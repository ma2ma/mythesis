\ctitle{基于\smarkdown和\LaTeX的高校文档云服务平台}
\cdepartment[计算机]{计算机科学与技术系}
\cmajor{计算机科学与技术}
\cauthor{马胜}
\csupervisor{许林英\hspace{4mm}副教授}
\ccosupervisor{李文兰 教授}
% 日期自动生成,如果你要自己写就改这个cdate
%\cdate{\CJKdigits{\the\year}年\CJKnumber{\the\month}月}
\etitle{A Study on Subspace Clustering Algorithm for High-dimensional Data} 
\edegree{master of Engineering} 
\emajor{Computer Science and Technology} 
\eauthor{Ma Sheng} 
\esupervisor{Professor Xu LinYin} 
\eassosupervisor{Li WenLan} 
% 这个日期也会自动生成,你要改么?
% \edate{December, 2005}

% 定义中英文摘要和关键字
\begin{cabstract}
近年来,随着web2.0网站流行和云计算模式逐渐发展,个人文档的写作,保存,和分享的模式也发生了很大的变化,越来越多的人喜欢在网上写文章,保存文档,和分享文档。随之也涌现了许多流行的网络云服务,比如github,markdown语法, 云笔记, 网盘等。较之比较流行的社交平台(faceboob,twitter等)这些服务更注重个人文档的建立,保存和分享,给我们的个人文档管理提供了极大地方便。

但在当今高校,由于网络和保密等诸多因素限制,这些便捷的文档服务平台并没有被广泛应用。学生和老师写论文和保存文档依旧以老旧的本地微机模式为主。以微软的office软件为主要的编辑工具,个人电脑为主要保存介质,文档的分享也多采用电子邮件这样的原始的方式。给教学和科研带来了诸多不便。

本服务平台的设计和实现即致力于改善现有高校的文档管理模式。给高校老师和学生提供一个高效率,高可用性的文档管理平台。平台将提供文档编辑,格式生成,文档保存,文档分发,版本控制,成果认证等许多有助于改善个人文档管理的功能。我相信使用该平台将给当今高校科研和教学提供极大帮助。使老师和学生专注于本专业的研究与学习。而不会把时间更多的浪费在排版格式和文档管理上。
\end{cabstract}

\ckeywords{\smarkdown, \LaTeX, 模版, 云笔记, restful, golang,mongodb}

\begin{eabstract} 
   An abstract of a dissertation is a summary and extraction of research work
   and contributions. Included in an abstract should be description of research
   topic and research objective, brief introduction to methodology and research
   process, and summarization of conclusion and contributions of the
   research. An abstract should be characterized by independence and clarity and
   carry identical information with the dissertation. It should be such that the
   general idea and major contributions of the dissertation are conveyed without
   reading the dissertation. 

   An abstract should be concise and to the point. It is a misunderstanding to
   make an abstract an outline of the dissertation and words ``the first
   chapter'', ``the second chapter'' and the like should be avoided in the
   abstract.

   Key words are terms used in a dissertation for indexing, reflecting core
   information of the dissertation. An abstract may contain a maximum of 5 key
   words, with semi-colons used in between to separate one another.
\end{eabstract}

\ekeywords{markdown, \LaTeX, template, cloudnote, restful, golang}
\declaretitle{独创性声明}
\declarecontent{
本人声明所呈交的学位论文是本人在导师指导下进行的研究工作和取得的研究成果,除了文中特别加以标注和致谢之处外,论文中不包含其他人已经发表或撰写过的研究成果,也不包含为获得 {\underline{\kai\textbf{~天津大学~}}}或其他教育机构的学位或证书而使用过的材料。与我一同工作的同志对本研究所做的任何贡献均已在论文中作了明确的说明并表示了谢意。
}
\authorizationtitle{学位论文版权使用授权书}
\authorizationcontent{
本学位论文作者完全了解{\underline{\kai\textbf{~天津大学~}}}有关保留、使用学位论文的规定。特授权{\underline{\kai\textbf{~天津大学~}}} 可以将学位论文的全部或部分内容编入有关数据库进行检索,并采用影印、缩印或扫描等复制手段保存、汇编以供查阅和借阅。同意学校向国家有关部门或机构送交论文的复印件和磁盘。
}
\authorizationadd{(保密的学位论文在解密后适用本授权说明)}
\authorsigncap{学位论文作者签名:}
\supervisorsigncap{导师签名:}
\signdatecap{签字日期:}