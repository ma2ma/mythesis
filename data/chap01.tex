
\chapter{概述}
\label{chap:intro}

\section{设计理念}
\label{sec:dongji}

本系统的设计与实现是天津大学图书馆建立学术科研和教育教学提供支撑系统体系的一部分。原本要设计的是一个高校机构库管理系统。主要功能是收集并整理本校老师和博硕士的科研成果,并组织分类后通过统一的发布平台进行展示。要实现的功能对本校科研水平展示和对本校学术资源的收集。但是通过对厦门大学和中科院等国内知名机构库系统的调研后发现:
\begin{enumerate}
\item 老式机构库系统从访问量上并没有达到预想的效果。由于现在更多的人习惯使用搜索引擎,博客等渠道作为获取信息的主要途径。所以目前国内主要的高校机构库平台访问量都不理想。基本上属于 “信息孤岛” 状态。
\item 老式机构库从内容的管理比较繁琐,信息实时性差,更新费时费力。由于老式机构库从结构上依然属于信息资源管理系统(erp),由于管理类系统的操作界面相对复杂,加之信息录入工作比较繁琐,所以大多老师学生不愿意使用。所以其主要的建库和信息维护工作还是需要由,机构库的管理部门(比如图书馆)来完成,而这些部门需要维护的信息来源却是学校里的教授或者学生。这就出现了信息多次传递的问题。带来的后果自然是管理部门费时费力,教授和学生不满意。
\item 老式机构库从系统实现上国内还是以dspace,Eprints,超星等平台为主,其内容的实现机理也是基本上面向数据存储的。这就给信息的结构化扩展带来了很多问题。
\end{enumerate}
除了以上列出的几个方面的问题以外,老式机构库追究其问题的根源是:这些系统是面向信息收集和信息管理部门而设计的。其更多的考虑了信息如何收集和展示。而忽略了信息来源(即老师和学生)在系统中的作用。管理部门更像实在自说自话,不断的向一个很少有人访问的门户网站上添加内容。而更加悲剧的是甚至这些信息的创作者有时都不知道还有这么一个平台上有他们写的东西。这种有着强烈web 1.0味道的应用在当今这个时代是必然会被淘汰的。

由于考虑到以上诸多问题,所以本系统开发之初即转换了设计理念。放弃了机构库面向信息收集、管理、发布为主要目标的设计,采用了面向信息创作者,为信息创作者提供便捷易用的个人文档管理平台为主要目标的设计理念:
\begin{itemize}
\item 系统主要的使用者是本校教师和学生。他们是系统主要的信息提供者
\item 以帮助教师和学生提高工作效率为首要目标。系统功能以实际出发帮助他们解决个人文档管理中的实际问题。
\item 由于教师和学生不是专职的信息管理人员,所以要系统必须提供简单易用的使用界面,上手使用必须无需任何额外的学习过程。
\item 系统信息组织不再以集中式的信息收集为主,而是把系统信息以个人文档的形式保存在每个人的私有空间下,每个用户决定这些文档的公开程度和公开范围。
\item 由于系统是面向个人的,所以系统不仅要成为个人文档的存放平台。还要成文个人文档的建立平台。所以系统的功能也更多的偏向于‘纯文本编辑’功能。
\item 如果本系统可以被广大师生接受并大量使用。那么把个人空间内公开的文档整理并展示将不是非常困难的事情,可以说是顺理成章的事情。所以本系统具有老式机构管理库绝大多数功能。
\item 信息管理部门和学校其他职能部门虽然不再是管理信息的主要角色。但是在本系统中他们可以为教师和学生提供类似“信息补全” “成果认证” 等一系列的服务。让用户可以更好的使用系统。
\end{itemize}
通过上面的描述,可以看出本系统的主要设计理念是“面向用户”而非“面向系统”。这一点理念上的变化有点像web 1.0到web 2.0的变化。由门户网站到博客、微博、个人空间的变化。如果用一句话来概括本系统的设计理念的话,应该是“如果用户可以在我们的帮助下把自己的文档和成果管理好,那么我们就可以帮助他们更好的提升并展示自己”。还需要解释的是:由于系统设计理念的不同,所以这套系统在命名上也不再沿用“机构管理库”这样的名词。从功能上主要以管理个人文档和成果为主,所以暂时以“个人学术文档服务平台”命名。

刚才提到系统要以帮助教师和学生解决个人文档管理中遇到的问题为主要目标。h那么到底在现在高校中老师和同学们在教学科研和学习中都有那些问题急需解决呢?我们将在下一个小节来汇总一下。

\section{要解决的问题}
\label{sec:question}

上一节提到本系统的设计理念是:帮助每一个用户解决个人文档管理中遇到的问题为主,收集和展示文档资源为辅。现在我们就来分析一下当今高校教师和学生在管理个人文档方面都有哪些问题:
\begin{enumerate}
\item 无论是学术成果的期刊论文、学生的毕业论文、个人编著的书籍、项目的申报和评审材料、技术报告书、教学演示文稿等学术性文档,还是个人的工作笔记、项目日程计划、工作日记等普通文档。在当今高校中,制作这些文档的主要工具是microsoft的office系列软件。office系列软件有“所见即所得”、“易于上手”等诸多优点,在目前社会(起码在中国)上也有相当大的用户群和相当高的普及率。
\end{enumerate}