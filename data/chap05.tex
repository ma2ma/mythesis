
\chapter{总结和展望}
\label{chap:sumandprospect}

\section{总结}
\label{sec:zongjie}

本论文首先综述了Smarkdown标记语言、LATEX和社交的网络等技术 ,Smarkdown标记语言简化了用户撰写文档的方式,LATEX技术提供了丰富的排版功能,云端存储的方式提供了用户文档的存储,社交网络技术解决了文档的分享。集成以上技术,本文提出了一个文档管理与服务的应用平台的需求和解决方案。

该解决方案采用Ruby on Rails开发框架作为基本开发平台,采用非关系型数据库MongoDB和Mysql数据库联合作为后台数据存储,前端界面采用Bootstrap页面框架,并配合Liquid模版语言配置灵活的主题模版,以提供友好的用户界面。

具体实现中,本平台设计了一个扩展的Smarkdown标记语言,通过对标准的markdown标记语言语法进行扩展,以满足撰写科技文档的使用需要。并独立的实现了扩展语法后的smarkdown语言的编译器。并且设计了一个开放的RESTful应用程序接口, 可方便地与其他系统无缝对接,而且诶实现了简单的官方移动客户端和PC客户端。

目前,高校文档云服务平台的关键功能已经完成,并完成了基本系统的部署,相关的用户试用的结果表明该平台已经达到了设计的目标。

\section{展望}
\label{sec:zhanwang}

实现现有的系统功能后,本系统可以说基本上实现了要实现的目标,完成了设计中所有的功能,也基本解决了高校用户在文档管理中遇到的诸多问题。系统也拥有了一定数量的用户。并积累了一定量的文档。那么针对这些用户和文档进行数据分析,将是系统下一步要做的主要工作。通过对大量的数据和应用很好的分析算法,可以获得很多有趣的分析结果,比如某些学科的老师感兴趣的文章有哪些,某学校的某位老师是否本领域有很到的关注度等等,这些信息对于学校对人才的评定和学科的热门程度的判断都具有非凡的意义。而且通过数据分析,系统也可以实现更加智能的功能(系统推荐感兴趣的文档等)。除此以外,系统还将增加对R语言\footnote{一门热门的数据统计语言,在国外,被认为是高校学生必学的工具语言}的支持。也将增加项目管理(类Basecamp)和 GTD(个人事务管理)等功能,方便用户管理日常工作和生活。

可以看到,系统目前实现的功能可以说只是一个''雏形'',数据分析,和相应的扩展功能完成后才真正可以说是''完型''。

当然系统要真正实现理想中的状态,不仅需要技术上完美的实现,有时也需要积极的推广,学校各职能部门大力支持,和广大师生的积极相应。但是有一点本人始终坚信: 只要系统的功能完善,界面友好,运行稳定,简单易用。那么,无论是刚刚走进大学校园的新生还是在教学科研第一线工作多年的专家教授,都会逐渐习惯并喜欢本系统为文档管理模式带来的改变,并成为本系统忠实的用户。
