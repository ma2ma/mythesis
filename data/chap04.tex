
\chapter{系统实现}
\label{chap:achieve}

上文介绍了本系统的系统架构和逻辑功能。本章我们将进行部分的系统功能实现。由于本系统结构庞大,功能复杂,涉及许多不同体系的技术,所以限于篇幅有限,我们并不打算把系统涉及到的全部实现细节一一介绍。而只介绍其中最核心的,和最具有代表性的部分。

由于基于\smarkdown的文档撰写是本系统区别其他网络文档管理平台的最大特点。而且由于\smarkdown语法是本文作者在Markdown语法基础上扩展而来。所以介绍\smarkdown语法和它如何被渲染展示和如何被转换成\LaTeX,再进一步转换成最终的PDF文档将是我们本章介绍的重点。

但是在介绍这部分之前,我们还是先来了解一下本系统主要涉及的技术,以及用到的编程语言,框架,和组件。

\section{系统涉及的实现技术}
\label{sec:language}

本系统要实现的功能繁多,而且要运行的平台也多样\footnote{windows应用,ios app,andriod app,web应用等}。最粗略的统计,主要包括:web service服务端、数据库、web前端、windows客户端,移动客户端几个部分。下面分别进行介绍:

\subsection{web service服务端}
\label{sec:webservice}

上文介绍过,本系统基本架构为restful的web服务。目前支持restful的编程语言和成型的开源架构很多,比如Python、java、.net等等。本系统选择的后端编程语言为 GO语言,也许很多读者没有听说过这门语言。这也难怪,它的确是一门新兴的语言,发布的时间也并不长。但是看完下面的简单介绍,相信任何人都不会小看它。

%%%%%介绍vgo语言%%%%%%

的确,GO语言功能强大,语法简单,而且适合当今互联网和多核盛行的环境,但是它却有着一个必须要面对的缺点,那就是它真的太“年轻”了,这意味这用它设计实现的可以利用的代码库少的可怜,而且质量不高。很多很基础的功能,使用其他语言可以直接调用一个成熟的库,用几个简单的函数调用就可以解决,但是使用GO语言就要自己去编写这些功能。这无疑给系统的实现带来麻烦。之所以最终选择它,其实也是一个权衡利弊的过程,也有本文作者个人喜好的原因\footnote{本人使用java,.net,c++,c做过很多不同的软件,对于Python和perl等脚本语言也比较熟悉,个人比较喜欢的编程语言是c和perl,一直梦想着有一个编程语言能像c一样简单强大而又可以方便的编写大型软件。我想go就是我一直寻找的。}